% Options for packages loaded elsewhere
\PassOptionsToPackage{unicode}{hyperref}
\PassOptionsToPackage{hyphens}{url}
%
\documentclass[
]{article}
\usepackage{lmodern}
\usepackage{amssymb,amsmath}
\usepackage{ifxetex,ifluatex}
\ifnum 0\ifxetex 1\fi\ifluatex 1\fi=0 % if pdftex
  \usepackage[T1]{fontenc}
  \usepackage[utf8]{inputenc}
  \usepackage{textcomp} % provide euro and other symbols
\else % if luatex or xetex
  \usepackage{unicode-math}
  \defaultfontfeatures{Scale=MatchLowercase}
  \defaultfontfeatures[\rmfamily]{Ligatures=TeX,Scale=1}
\fi
% Use upquote if available, for straight quotes in verbatim environments
\IfFileExists{upquote.sty}{\usepackage{upquote}}{}
\IfFileExists{microtype.sty}{% use microtype if available
  \usepackage[]{microtype}
  \UseMicrotypeSet[protrusion]{basicmath} % disable protrusion for tt fonts
}{}
\makeatletter
\@ifundefined{KOMAClassName}{% if non-KOMA class
  \IfFileExists{parskip.sty}{%
    \usepackage{parskip}
  }{% else
    \setlength{\parindent}{0pt}
    \setlength{\parskip}{6pt plus 2pt minus 1pt}}
}{% if KOMA class
  \KOMAoptions{parskip=half}}
\makeatother
\usepackage{xcolor}
\IfFileExists{xurl.sty}{\usepackage{xurl}}{} % add URL line breaks if available
\IfFileExists{bookmark.sty}{\usepackage{bookmark}}{\usepackage{hyperref}}
\hypersetup{
  pdftitle={Ejercicio final. Desafío optativo.},
  pdfauthor={Capacitaciones R en DET},
  hidelinks,
  pdfcreator={LaTeX via pandoc}}
\urlstyle{same} % disable monospaced font for URLs
\usepackage[margin=1in]{geometry}
\usepackage{graphicx,grffile}
\makeatletter
\def\maxwidth{\ifdim\Gin@nat@width>\linewidth\linewidth\else\Gin@nat@width\fi}
\def\maxheight{\ifdim\Gin@nat@height>\textheight\textheight\else\Gin@nat@height\fi}
\makeatother
% Scale images if necessary, so that they will not overflow the page
% margins by default, and it is still possible to overwrite the defaults
% using explicit options in \includegraphics[width, height, ...]{}
\setkeys{Gin}{width=\maxwidth,height=\maxheight,keepaspectratio}
% Set default figure placement to htbp
\makeatletter
\def\fps@figure{htbp}
\makeatother
\setlength{\emergencystretch}{3em} % prevent overfull lines
\providecommand{\tightlist}{%
  \setlength{\itemsep}{0pt}\setlength{\parskip}{0pt}}
\setcounter{secnumdepth}{-\maxdimen} % remove section numbering
\usepackage[fontsize=12pt]{scrextend}

\title{Ejercicio final. Desafío optativo.}
\author{Capacitaciones R en DET}
\date{12-01-2020}

\begin{document}
\maketitle

\hypertarget{objetivo}{%
\subsection{Objetivo}\label{objetivo}}

Automatizar en lenguaje R algún procesamiento o análisis realizado
periódicamente al interior de la institución.

\hypertarget{indicaciones-generales}{%
\subsection{Indicaciones generales}\label{indicaciones-generales}}

Esta última tarea de la capacitación en R consiste en un ``desafío'', en
donde cada compañero/a debe ``automatizar'', pasando a lenguaje R o
RMarkdown, algún proceso o análisis que habitualmente hace en otro
software como Excel o SPSS. La idea es volver más eficiente un proceso,
reduciendo sus tiempos de ejecución, disminuyendo sus posibles fuentes
de error y volviéndolo reproducible. Se espera que el código que
elaboren, al mediano y largo plazo, les simplifique la vida.

Se recomienda trabajar con Rproject, con directorios relativos, con
Rmarkdown y con un sistema de carpetas ordenado según el
\href{https://capacitacionesdet.github.io/sesion_2/\#9}{protocolo IPO}.

\hypertarget{ejemplos-e-ideas}{%
\subsection{Ejemplos e ideas}\label{ejemplos-e-ideas}}

Los procesos a automatizar pueden ir desde la construcción de tabulados
hasta la elaboración de boletínes internos o externos, pasando por
aplicaciones de validadores, auditorías, lecturas de coyuntuas y otros
reportes. Básicamente todas las tareas que realizamos habitualmente en
la institución pueden ser formalizadas y traspasadas a lenguaje R.

Al interior de la institución ya se han hecho esfuerzos en esta línea:

\begin{itemize}
\item
  Aplicación de validadores en Encuesta Mensual IR-ICMO, que crea
  automáticamente archivos Excel para auditar manualmente.
\item
  Elaboración de presentaciones de lecturas de coyunturas internas.
\item
  Automatización de boletines regionales, nacionales y complementarios
  (Boletín COVID-19 Módulo IR-ICMO, por ejemplo).
\end{itemize}

\hypertarget{apoyo-en-el-desafuxedo}{%
\subsection{Apoyo en el desafío}\label{apoyo-en-el-desafuxedo}}

Antes de comenzar a elaborar el ejercicio, evaluar su pertinencia y
factibilidad con los capacitadores. Si gusta, puede enviar una pequeña
propuesta de lo que realizará entre el 12 y el 18 de enero de 2021, para
que esta sea visada por el equipo.

Para el desarrollo de los ejercicios se pueden consultar las
presentaciones vistas en clases en la página web del curso
\url{https://capacitacionesdet.github.io/}, así como cualquier otro
material de internet que resulte pertinente. Particularmente útil y
novedoso puede resultar el material complementario
\href{https://capacitacionesdet.github.io/bonus1_openxlsx/\#1}{``Exportar
archivos desde R a Excel''}.

Cualquier duda o problema no dude en contactar al equipo a cargo de las
capacitaciones. En esta línea, los días martes 19 de enero, 26 de enero
y 02 de febrero se reservará el bloque 09:30 - 11:00 como espacio de
consultas, en el cuál se podrán revisar avances mediante videollamada en
teams.

\hypertarget{evaluaciuxf3n-y-nota-de-muxe9rito}{%
\subsection{Evaluación y nota de
mérito}\label{evaluaciuxf3n-y-nota-de-muxe9rito}}

Quien entregue el ejercicio antes de la fecha límite (15 de febrero) y,
este sea evaluado positivamente por el equipo capacitador, obtendrá una
nota de mérito que acreditará el haber sido capacitado en R, el haber
cumplido con todas las tareas y ejercicios planificados, y el haber
innovado en el desarrollo de sus labores.

Enviar ejercicio final como archivo comprimido, con nombre y apellido de
quien hizo la tarea, a los correos \texttt{nicolas.ratto@ine.cl} y
\texttt{gonzalo.franetovic@ine.cl}. Si la carpeta pesa mucho se puede
compartir mediante onedrive, dándo acceso a los correos mencionados. La
carpeta del ejercicio final debe tener a lo menos un archivo .R o .Rmd
con la explicación general del proceso y el código que lo hace posible.

\end{document}
